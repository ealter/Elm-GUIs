\documentclass{article}

\usepackage{tikz}
\usetikzlibrary{arrows}
\usepackage{graphicx}
\usepackage{enumerate}
\usepackage{fancyhdr}

\usepackage{hyperref}
\setlength{\headheight}{15pt}
\pagestyle{fancyplain}

\setlength{\parindent}{0pt}
\setlength{\parskip}{2ex}

\tikzstyle{int}=[draw, minimum size=4em]
\tikzstyle{init} = [pin edge={to-,thin,black}]
\lhead{Dynamic Hover Detection: Creating GUIs in Elm}
\rhead{Goldstein \& Alter}

\makeatletter
\renewcommand{\maketitle}{ % Customize the title - do not edit title and author name here, see the TITLE block below
    \begin{flushright} % Right align
        {\LARGE\@title} % Increase the font size of the title

        \vspace{40pt} % Some vertical space between the title and author name

        {\large\@author} % Author name

        \vspace{30pt} % Some vertical space between the author block and abstract
    \end{flushright}
}

% This begins the document
\title{\textbf{Dynamic Hover Detection:\\Creating GUIs in Elm}}
\author{Max Goldstein \& Eliot Alter}

\begin{document}
\thispagestyle{empty} % No header on first page
\maketitle

\begin{abstract}
The language Elm was designed to be a practical application of
Functional Reactive Programming (FRP) to create Graphical User
Interfaces (GUIs). We attempted to create a menu bar as commonly seen on
desktop operating systems, and quickly discovered an apparent
incompatibility between Elm's mouse-hover detection primitive and the
constraints of the language. This concern only surfaces when dealing
with highly dynamic data. We present a non-obvious technique to address
the issue without modifying the Elm compiler. It generalizes to other
functions in Elm's \texttt{Graphics.Input} library, which includes
(besides hover detection) GUI mainstays such as buttons, checkboxes, and
text fields. We also contribute other techniques for the representation and
display of menus in Elm, and contrast our work with an existing Elm web
application.
\end{abstract}

\section{Introduction}\label{introduction}

Elm was introduced in March 2012 in Evan Czaplicki's senior thesis.
\cite{elm-thesis} He and his adviser, Stephen Chong, published a formal
description of Elm's semantics at PLDI 2013. \cite{CzaplickiC13} Both papers are
comprehensive overviews of Elm, and additionally provide excellent literature
reviews of previous FRP GUI endeavors, of which there are several. Elm compiles
down to JavaScript to run in the browser, making it remarkably portable.

Elm was designed to make implementing GUIs easier, so we decided to try to
implement menus as an example GUI. The particular design places the top-level
menu at the top of the screen, similar to Mac OS X and many Linux distributions,
with selections coming down from the top. Menus extend when the top-level item
is hovered upon, and remain extended while the mouse hovers over any item in the
menu. Therefore it is necessary to know hover information about each menu item.
This time-varying information is also used to detect selections upon click and
highlight the moused-over item. It is simple to do this when the hover-detecting
area is constant. This paper describes the much more difficult task of managing
time-varying hover information about time-varying areas.

There are two features of Elm we are deliberately avoiding. First is the
extensive raster drawing library, \texttt{Graphics.Collage}. Dynamic hover
detection is not problematic when using this library because it can be done
purely using geometric collision detection. However, if we used this library, our
GUI would be a single raster animation and not a DOM tree. Secondly, the
\texttt{Graphics.Input} library contains wrappers around HTML checkboxes and
dropdowns. We rejected creating GUIs with these and wanted to pursue something
more general. This led to the choice of an interface typically found in the
operating system rather than the browser. We do refer to the native GUI
constructs to show how the technique we develop for hovering generalizes to
them, as their API is similar to the one for hover detection.

It is difficult for the authors to assess what level of knowledge should
be assumed on the part of the reader. Firstly, readers will range from
Elm's creators and experienced users, who are already familiar with its
restrictions and standard libraries, to those with no FRP background,
who need an introduction to signals. Secondly, while we have found no
prior discussion on either the problem we identify nor its solution, we
cannot know for certain that either are novel. We continue in the hopes
of presenting new and non-trivial techniques to the Elm community.

We contribute:

\begin{itemize}
\itemsep1pt\parskip0pt\parsep0pt
\item
  The identification of an apparent limitation in Elm's hover detection
  library, solved by a non-trivial usage pattern that does not require
  modifications to the Elm compiler or runtime, and that generalizes to
  other functions in \texttt{Graphics.Input}.
\item
  An implementation of desktop-style menu in Elm, which incorporates
  several noteworthy ``tricks'' as well as general style best practices.
\item
  An analysis of TodoFRP, the current state-of-the-art in dynamic Elm
  GUIs. We demonstrate how it operates in the absence of our technique,
  and how it could operate in its presence.
\end{itemize}

Section \ref{elm-for-functional-programmers} introduces Elm, signals, state, and
the prohibition on signals of signals. It is targeted to readers familiar with
functional programming but not FRP, and may be skipped by those already
comfortable with Elm.  Section \ref{detecting-hover-information} presents Elm's
hover detection API for DOM elements, why it initially appears inadequate, and
how it can be extended to meet our needs. Section \ref{implementing-menus}
explains our menu implementation in detail. Section \ref{related-work-todofrp}
analyzes TodoFRP. Section \ref{conclusion} concludes with a notice to the Elm
community.

\section{Elm for Functional
Programmers}\label{elm-for-functional-programmers}

A typical functional program is \emph{transformative}: all input is
available at the start of execution, and after a hopefully finite amount
of time the program terminates with some output. In contrast, Elm
programs are \emph{reactive}: not all input is available immediately and
the program may indefinitely adjust output with each input. In simple
programs, the inputs at a given time fully determine the output. More
complex programs will take advantage of Elm's ability to remember state.

\subsection{Signals: Time-varying
values}\label{signals-time-varying-values}

A time-varying value of a polymorphic type \texttt{a} is represented by
\texttt{Signal a}. For example, the term \texttt{constant 150} has type
\texttt{Signal Int}. The combinator \texttt{constant} creates a signal
whose value never changes. A more interesting signal is the primitive
\texttt{Window.dimensions : Signal (Int, Int)}. This signal represents
the browser window size and updates whenever it is resized. Signals are
asynchronous in that they update at no set time, just as the window may
remain the same size indefinitely. Signals update in discrete events, but
are continuous in the sense that they are always defined.

The function \texttt{lift} allows us to execute a pure function on a
signal of inputs, producing a signal of outputs. (Although lifting is a
general functional concept, in Elm it has only this meaning.)

\begin{verbatim}
lift : (a -> b) -> Signal a -> Signal b
\end{verbatim}

For example, we can multiply the width and height of the window together
to find its area.

\begin{verbatim}
area : Signal Int
area = lift (uncurry (*)) Window.dimensions
\end{verbatim}

In this case, the lifted function is multiplication, uncurried as to
operate on pairs. As the Window library exposes width and height as both
a pair and individually, we can also write
\texttt{area2 = lift2 (*) Window.width Window.height}.

We can print the current area to the screen with \texttt{main = lift asText
area}. The primitive \texttt{asText : a -> Element} renders almost
anything into an Element, which represents a DOM element.\footnote{Those
    following along in an Elm compiler, such as the one available at
    \texttt{elm-lang.org/try}, should add \texttt{import Window} to the top of
the file.}

The signals in an Elm program can be thought of as a Directed Acyclic
Graph (DAG). Many signals depend on other signals for their output. For
example, the \texttt{area} signal depends on \texttt{Window.dimensions}
signal. Similarly, the \texttt{area2} signal depends on both the values
of the \texttt{Window.width} and the \texttt{Window.height} signals.
When an event is fired on a signal, it propagates down the DAG to all
signals who depend on that signal and the outputs of those signals are
reevaluated, recursively.

\subsection{Remembering State}\label{remembering-state}

Signals can remember state by using the \texttt{foldp} combinator.
Familiar list folds apply a binary operation of an element and an
accumulator to produce a new accumulator, over each list element in
sequence. Folding from the \emph{past} operates on all of the values of
a signal as they occur and produces a signal of the accumulator.

\begin{verbatim}
foldr : (a -> b -> b) -> b -> [a]      -> b
foldp : (a -> b -> b) -> b -> Signal a -> Signal b
\end{verbatim}

When the event signal updates, a pure function is called with the new
event and the old accumulator (a default is supplied), producing a new
accumulator that is the new value of the output signal. For example,
\texttt{foldp max 0 Window.width} is a signal of the maximum width ever
obtained by the window. With \texttt{foldp}, it is possible to create
signals that depend on every event to ever occur on a signal. However,
most folded functions do not store every value explicitly (cons is an
exception) and space can be saved by remembering only the accumulator.

\subsection{No Signals of Signals}\label{no-signals-of-signals}

The ability to create a signal dependent on past state has a potentially
disastrous implication for space performance. Czaplicki and Chong explain:

\begin{quote}
Intuitively, if we have signals of signals, then after a program has executed
for, say 10 minutes, we might create a signal that (through the use of
\texttt{foldp}) depends on the history of an input signal, say
\texttt{Window.width}. \cite{CzaplickiC13}
\end{quote}

That is, we can define a clock that measures the amount of time since
program execution began:

\begin{verbatim}
clock : Signal Time
clock = foldp (+) 0 (fps 10)
\end{verbatim}

The \texttt{fps n} combinator produces a signal of the time elapsed
since its last event, updated \texttt{n} times a second. We sum these
deltas starting with zero. Then we create a function from a time to one
of two signals, one of which is trivial and one of which depends on the
history of \texttt{Window.width}:

\begin{verbatim}
switcher : Time -> Signal Int
switcher t = if t < 10*minute
             then constant 0
             else foldp max 0 Window.width
\end{verbatim}

We could, in theory, create

\begin{verbatim}
switched : Signal (Signal Int)
switched = lift switched clock
\end{verbatim}

Why is this problematic? Czaplicki and Chong continue,

\begin{quote}
To compute the current value of this signal, should we use the entire
history of \texttt{Window.width}? But that would require saving all
history of \texttt{Window.width} from the beginning of execution, even
though we do not know whether the history will be needed later.
Alternatively, we could compute the current value of the signal just
using the current and new values of \texttt{Window.width} (i.e.,
ignoring the history). But this would allow the possibility of having
two identically defined signals that have different values, based on
when they were created. We avoid these issues by ruling out signals of
signals. \cite{CzaplickiC13}
\end{quote}

Indeed, Elm's type system is principled on disallowing signals of signals.
Programs may not create in the future signals that depend on the past. Were they
allowed to, the system must either remember indefinitely every event to ever
occur on any signal, so that a newly created signal can use them, or tolerate
signals that vary based only on when they were created, losing referential
transparency.

To a reader familiar with Haskell, this means signals are functors (and
in fact applicative functors) but not monads, as monads support the
following operation:

\begin{verbatim}
join :: Monad m => m (m a) -> m a
\end{verbatim}

Such an operation for signals would condense a
\texttt{Signal (Signal a)} into a mere \texttt{Signal a}, but it cannot
exist in general.

\section{Detecting Hover Information}\label{detecting-hover-information}
A cursory inspection of \texttt{Graphics.Input} leads to the conclusion that
menus cannot be implemented as we had hoped. The remarkable solution generalizes
from detecting hover information to other GUI functions in the library.

\subsection{A Na\"{i}ve Menu}\label{a-naive-menu}

Elm's \texttt{Graphics.Input} library provides a function to obtain the hover
information from an Element.

\texttt{hoverable : Element -> (Element, Signal Bool)}

The returned Element is visually identical to the argument, but now detects
hover information. The signal of Booleans reflects the hover status of
the this Element, not the original. This function works well when the
Element is pure (not a signal). For example, if the top-level menu items are
known statically and never change, they can be rendered as pure Elements with
this function. We can also create a function, perhaps of type \texttt{Bool ->
Element}, which we can lift onto the hover data to display the submenu only on
mouseover. The result is a value of type \texttt{Signal Element}. If we try to
lift \texttt{hoverable} to accept this value, we get

\texttt{lift hoverable : Signal Element -> Signal (Element, Signal Bool)}

Although it may not appear to be so at first, the result type is a signal of
signals. It is possible to transform the result into \texttt{Signal Signal
(Element, Bool)}. A more mentally convenient type, though impossible to
obtain,\footnote{We explain why in Section \ref{menu-representations}.}
is \texttt{(Signal Element, Signal Signal Bool)}. There does not exist a general
join function to operate on the \texttt{Signal Signal Bool}.

This is rather unfortunate, as we need to detect hover information on a dynamic
element for a number of reasons. First, after a menu is extended by hovering
over its parent, it must remain so as long as the mouse hovers over the menu
itself. Without this information, the menu will disappear as soon as we try to
mouse over it. Secondly, if we wish the indicate the currently hovered-over item
to the user by visually highlighting it, the element must be dynamic. Thirdly,
we do not wish to require that menu labels be string literals in the source
code, which would be necessary if we could not use signals of elements. Rather,
they may be retrieved or generated in some other manner, or they may change as
the user switches among applications. Finally, in order to determine when a user
has clicked on an Element so that the system may respond, we must know which one
is being hovered over.

\subsection{Dynamic Hovering}
However, it is possible to implement the following function:

\texttt{hoverableJoin: Signal Element -> (Signal Element, Signal Bool)}

Comparing its type to the mentally convenient type of \texttt{lift hoverable},
we see it is identical except that the Boolean value is a signal, not a signal
of signals. That is, it has been joined.\footnote{While the name
    \texttt{hoverableJoin} reinforces this fact for our paper, we do not
recommend it for a user-facing API.} It is implemented using the more general
primitive \texttt{hoverables} (note the plural), of the following type:

\begin{verbatim}
hoverables : a -> { events : Signal a,
                    hoverable : (Bool -> a) -> Element -> Element }
\end{verbatim}

The polymorphic \texttt{a} type can serve as an identifier. The first
value supplies the default value of \texttt{events} (signals must always
be defined and so a default value is required). The returned record
includes the \texttt{events} signal and the \texttt{hoverable}
function, which in general may be applied multiple times so that
multiple elements report their hover information on \texttt{events}. The
\texttt{(Bool -> a)} is used to identify which Element
experienced the event; a common use is \texttt{a = (Int, Bool)} where
the integer identifies the Element and the Bool is the event.
Additionally, \texttt{hoverables} is used to implement
\texttt{hoverable}:

\begin{verbatim}
hoverable : Element -> (Element, Signal Bool)
hoverable elem =
    let pool = hoverables False
    in  (pool.hoverable id elem, pool.events)
\end{verbatim}

It ignores the polymorphism (\texttt{a = Bool}) and instead create a
Boolean signal that is originally false and use the identity function to
not alter the hoverable information. With a simple change, we can create
a function that acts on \texttt{Signal Element} instead of
\texttt{Element}:

\begin{verbatim}
hoverableJoin : Signal Element -> (Signal Element, Signal Bool)
hoverableJoin elem =
    let pool = hoverables False
    in (lift (pool.hoverable id) elem, pool.events)
\end{verbatim}

Notice that \texttt{pool.hoverable} is partially applied to \texttt{id}
purely, and then lifted on to the argument. This is possible,
ultimately, because \texttt{pool.hoverable} is pure.

Just how novel is this small, but hugely significant change? The
\href{http://docs.elm-lang.org/library/Graphics/Input.elm\#hoverables}{documentation}
for \texttt{hoverables} states that it allows users to ``create and
destroy elements dynamically and still detect hover information,'' but
gives no further indicators on how to do so. Though the Elm website is
full of examples, there are none for either \texttt{hoverable} or
\texttt{hoverables}. Moreover, in the
\href{https://groups.google.com/d/msg/elm-discuss/QgowLy5jdhA/CZQfjkbjMsEJ}{mailing
list post} that introduced these functions, Czaplicki said that
``\texttt{hoverables} is very low level, but the idea is that you can
build any kind of nicer abstraction on top of it.'' We have done just
that.

When a element changes, \texttt{hoverableJoin} attaches the new element to the
same hovering signal. If we were to display both the old and the new elements on
the screen, hovering over either would trigger the signal.

More dangerously, it becomes easy to create an infinite loop. Suppose an
Element shrinks on hover. Suppose the cursor hovers on the Element,
which is then replaced by a smaller Element, so the cursor is no longer
hovering on the Element. Then the original Element is put back, but is
now being hovered on! This condition manifests itself as flickering
between the two Elements. It is unavoidable in any language or system
that allows hover targets to change size in response to hover events.

\subsection{Generalized Joins on \texttt{Graphics.Input}}

It is not just hover detection that follows the paradigm of two primitives
with singular and plural names. Most of Elm's wrappers around HTML GUI
components do so as well, and have analogous types. They are therefore receptive
of the same join technique. For text-labeled buttons, press events are
represented by unit or a provided identifier:

\begin{verbatim}
 button : String -> (Element, Signal ())
 buttons : a -> { events : Signal a,
                  button : a -> String -> Element }
\end{verbatim}

If we wish to create a button whose string label varies, the types allow
the same exact same technique to be applied. (We have not tested whether
this approach works, but the types allow it.) The same goes for text
fields, which can be given text based on what else is known in the
program, and checkboxes. As currently implemented, checkboxes have
Boolean state so dynamic behavior has limited use. However, we can
imagine the ability to dynamically disable (``gray-out'') a checkbox in response
to other choices in a form.

Drop-down menus do not use the same paradigm as the rest; there is
\texttt{dropDown} but no \texttt{dropDowns}. This is unfortunate not
only because there seems to be no way to supply a
\texttt{Signal {[}String{]}}, but also because drop-downs are commonly used
in groups. Multiple menus reporting on one signal would be very
convenient. If the menu options were dynamic, it would be useful for the API to
indicate when an event occurred because the user made a selection vs.~when the
menu updated and the previously selected string was no longer available.

\section{Implementing Menus}\label{implementing-menus}

Having established the need for \texttt{hoverableJoin} and implementing it, we
now turn to stated task of creating menus.
\subsection{Menu Representations}\label{menu-representations}

A menu specification, the abstract model of a menu, can be thought of as an
instance of a tree.

\texttt{data Tree a = Tree a [a]}

The obvious type for a menu structure that may change is \texttt{Signal (Tree
String)}. It is easy enough to convert a value of this type to one of type
\texttt{Signal (Tree Element)}, but we cannot add hover detection to this tree.
We are able to map from \texttt{Signal Element -> (Signal Element,
Signal Bool)}, but not (in any useful way) from \texttt{Element ->
(Element, Bool)}. That is, because the tree is already a signal, we cannot
create signals inside it (without creating signals of signals). Hover
information requires a (distinct) signal at each item in the tree.

Instead of a signal of trees, we therefore use a tree of signals. Each
individual menu element can contain dynamic information, but the menu structure
must be static. We can, in practice, get around this restriction by creating a
tree that is larger than necessary and filling the unused nodes with empty
string, which our implementation handles appropriately.

Note that we can (and will) convert a tree of signals into a signal of trees,
but it is impossible to go the other way. When converting many signals into one,
the information on which signal changed is lost. The combinator \texttt{combine
    : {[}Signal a{]} -> Signal {[}a{]}} does this, and will be used below. It
    has no inverse because such a function would need to create information:
    which signal in the list updated? What if the list changes length? What if
    there is a repeat event, where the list does not change, which signals
    should propagate repeats? Similar arguments show why the mentally convenient
    type of \texttt{lift hoverable}, which turned a signal of pairs into a pair
    of signals, cannot actually be obtained.

\subsection{Menu Rendering}

Therefore, a menu is specified by a value of type \texttt{Tree (Signal String)}.
We can lift and map the combinator \texttt{plainText : String -> Element} to
obtain from a menu specification a value of type \texttt{Tree (Signal Element)}.
From there, we can use \texttt{hoverableJoin} to get the hover information of
each Element and create a value of type \texttt{Tree (Signal Element, Signal
Bool)} to represent our menus. \emph{Now} we convert (by structural recursion)
our tree of signals to a signal of trees, a value of type \texttt{Signal (Tree
(Element, Bool))}. We do the rest of our work with pure functions lifted to
operate on the tree, for example, our rendering function \texttt{renderMenu :
    Tree (Element, Bool) -> Element}. Similarly to limiting the use
    of the IO monad in Haskell code, it is advantageous to limit the use of
    signals in Elm code.

Because \texttt{hoverableJoin} was applied throughout the entire tree, it
contains every Element in every menu, and Booleans that indicate which ones to
display. Rather than omitting Elements associated with false, we replace these
hidden submenus with a spacer. A spacer is simply a blank rectangle with a
specified width and height (\texttt{spacer : Int -> Int
-> Element}). For our purposes, the spacer has the same width as
the Element, but a height of 1.\footnote{We were concerned a height of zero
would cause browser rendering issues.} We then switch between menu and the
spacer based on hover information. The spacer helps to align the submenu with
its parent element in a way a \texttt{Maybe Element} would not have done. By
replacing hidden elements with spacers, we are able to use the \texttt{flow}
primitive to position our elements instead of manually doing ``absolute''
positioning.

\begin{verbatim}
elemOrSpacer : (Element, Bool) -> Element
elemOrSpacer (elem, hover) =
    if hover then elem
    else spacer (widthOf elem) 1
\end{verbatim}

There was another complication involving the hover information. A submenu is
rendered to the screen if one of three conditions is met: the mouse is hovering
upon its parent, the mouse is hovering upon it, or the mouse is hovering upon
any of its descendants. While this definition allows recursive submenus to be
correctly rendered to the screen, it creates a race condition. As the mouse
moves from the parent to the its submenu, the child is often replaced with a
spacer before it can detect that the mouse is hovering over it. The results in
menus disappearing as the mouse changes which element it is hovering upon. To
fix this problem, we created the following function:

\begin{verbatim}
delayFalse : Signal Bool -> Signal Bool
delayFalse b = lift2 (||) b (delay millisecond b)
\end{verbatim}

Applying \texttt{delayFalse} makes a \texttt{Signal Bool} wait a millisecond to
transition from true to false. Unlike a typical delay, false-to-true transitions
still occur immediately. By applying this function to every hovering Boolean in
the menu structure, we give the mouse time to move from a menu Element to its
submenu before that submenu disappears.

A menu that does not detect mouse clicks is a fairly useless menu. There is no
primitive in Elm that can simply attach click detection to an element. To get
around this issue, we used a combination of \texttt{Mouse.clicks : Signal ()}
(which fires an event whenever the mouse clicks anywhere on the screen) and the
hoverable information to determine which element (if any) the mouse was hovering
over. (This implementation uses Elm's infix function application operator
\texttt{<|}, which is the same as the Haskell operator \texttt{\$}).

\begin{verbatim}
clicksFromHover : Signal Bool -> Signal ()
clicksFromHover hover =
    lift (\_ -> ())
    <| keepIf id False
    <| sampleOn Mouse.clicks hover
\end{verbatim}

For each menu element in the tree, we found its ``path'' (with type
\texttt{[Int]}) which denotes the list indexes needed to traverse to that
element. Combining this with the click information gave each element a signal of
its path that fired whenever the user clicked on that element. We then combined
all of those signals into one signal using the \texttt{merges} combinator
(\texttt{merges : [Signal a] -> Signal a}). The user of the menu could then use
this \texttt{Signal [Int]} to respond to mouse clicks. This is significantly
more flexible than returning a \texttt{Signal String} because a path is
guaranteed to be unique, whereas a menu string might have a duplicate.

Interestingly, we were able to implement menus without explicitly storing the
state of the menu, i.e. without using \texttt{foldp} (or Elm's implementation of
Arrowized FRP known as \texttt{Automaton}). Rather we use the current browser
DOM state to remember state. Instead of storing the menu currently being
displayed on the screen, we can derive it based on which Element (if any) the
mouse is currently hovering over. The elimination of state from our code allows
it to be much cleaner.

\section{Related Work: TodoFRP}\label{related-work-todofrp}

We examine \href{https://github.com/evancz/TodoFRP}{TodoFRP}, a simple
Elm web app created by Czapliki, to provide another example of the problem
solved by \texttt{hoverableJoin}. TodoFRP is the current state-of-the-art in highly
reactive Elm GUIs. It provides examples of different levels of
reactivity, a familiar context for Elm veterans, and a few dirty tricks
of its own.

TodoFRP presents the user with a text field asking, ``what needs to be
done?''. Entered TODO items become DOM elements, which can be deleted
with a ``x'' button, also a DOM element. The button is implemented using
the \texttt{Graphics.Input} function

\begin{verbatim}
customButtons : a -> { events : Signal a,
                       customButton : a -> Element -> Element -> Element -> Element }
\end{verbatim}

Notice the similarity with \texttt{hoverables}. Each call of
\texttt{customButton} provides the identifier event when the button is
clicked, and three (pure) Elements to display: one normally, one on
hover, and one on click. The result is an \texttt{Element}, not a
\texttt{Signal Element}, that nevertheless changes among those three in
response to the mouse. This is possible because the result Element's
dimensions are taken to be the maximum of the three inputs' dimensions.
Even if the Elements have different sizes, the resulting Element and
therefore the hover surface remains fixed in size. Although the same
join technique can be applied, we find it likely that it would not work
as intended.

In the case of TodoFRP, these three Elements are different colors of the ``x''
and the same for each TODO entry. The polymorphic \texttt{a}s are unique
identifiers (ascending integers) for each entry.

The TODO label Elements are dynamic and do not detect hover information.
The button Elements that do detect hover information are known
statically. With \texttt{hoverableJoin}, it becomes possible to
dismiss a TODO by clicking on its text label.

\section{Conclusion: To the Elm
Community}\label{conclusion}

It's true that we've used the \texttt{hoverables} function in a way that
it was (probably) never meant to be used, and there are some caveats
involved in doing so. Many small GUIs do not require dynamic,
hover-detecting elements. However, most large mouse-based GUIs do, and
creating them in Elm will necessarily encounter the obstacles we have
described.

We have implemented all of this without language modifications. However,
it is hoped that as the community becomes more familiar with functional
GUIs, new libraries are added to incorporate some of our tricks, or even
make them unnecessary. Elm's upcoming third-party library sharing system
looks to be an excellent opportunity to refine abstractions and idioms
for GUIs.

We present these techniques and analysis in the hopes they are of
service to the Elm and FRP communities. There are a number of obvious paths that
dead-end, and non-obvious paths that are fruitful. We hope this paper becomes
useful in Elm's goal of making GUIs simpler to implement and more robust
to use.

\section*{Acknowledgments}
We would like to thank Evan Czaplicki and Stephen Chong for creating
Elm, and the Elm community for growing it. We thank Norman Ramsey for
his guidance through functional programming, and our paper reviewer, Caroline
Marcks.

%Make sure to put a link to our GitHub
\bibliographystyle{alpha}
\bibliography{refs.bib}
\end{document}
